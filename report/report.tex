%----------------------------------------------------------------------------------------
%	PACKAGES AND DOCUMENT CONFIGURATIONS
%----------------------------------------------------------------------------------------

% !TeX spellcheck = it
\documentclass[12pt, a4paper, hidelinks]{article}

\usepackage{anyfontsize}
\usepackage{siunitx} % Provides the \SI{}{} and \si{} command for typesetting SI units
\usepackage{graphicx} % Required for the inclusion of images
\usepackage{subcaption}
%\usepackage{natbib} % Required to change bibliography style to APA
\usepackage{amsmath} % Required for some math elements
\usepackage[export]{adjustbox}
\usepackage{eurosym} % euro simbol
\usepackage{hyperref} % hyperlink
\usepackage[utf8]{inputenc}
\usepackage{bookmark}
\usepackage{float}
\usepackage{epigraph}
\usepackage{quoting}
\usepackage{newlfont}
\usepackage{color}
\usepackage{geometry}

\renewcommand{\labelenumi}{\alph{enumi}.} % Make numbering in the enumerate environment by letter rather than number (e.g. section 6)

\begin{document}
\begin{titlepage}

\begin{center}
{{\Large{\textsc{Alma Mater Studiorum $\cdot$ Universit\`a di Bologna}}}}
\rule[0.1cm]{15.8cm}{0.25mm}
\\\vspace{3mm}
%
%
{\Large{Dipartimento di Informatica - Scienza e Ingegneria\\
Artificial Intelligence}}


\end{center}

\vspace{20mm}

\begin{center}{
%
%
	{\LARGE{\textbf{Flatland Challenge}}}}
\end{center}

\vspace{15mm}

{\begin{center}
	 \large{Project Presentation}
\end{center}}

\vspace{32mm} \par \noindent

\begin{minipage}[t]{0.47\textwidth}
%
%
{\large{ Professor \vspace{2mm}\\{\textbf{Andrea Asperti}
}\\\\\\}}
\end{minipage}
%
\hfill
%
\begin{minipage}[t]{0.47\textwidth}\raggedleft{}{
{\large{ Students
\vspace{2mm}\\
\textbf{Alessandro Lombardi\\Fiorenzo Parascandolo} }}}
\end{minipage}

\vspace{31mm}

\begin{center}
Academic Year {2019/2020}
\end{center}

\end{titlepage}

{\tableofcontents}
\thispagestyle{empty}

\newpage
\setcounter{page}{1}

\section{Flatland}

\subsection{The classes RailAgentStatus and EnvAgent}

RailAgentStatus extends Python IntEnum and assumes the following values:
\begin{itemize}
	\item READY\_TO\_DEPART (0) the agent is not in the grid yet (position is None), the prediction is to stay at the starting position. If a MOVE\_* action is performed during this state it becomes ACTIVE\@.
	\item ACTIVE (1) the agent is in the grid (position is not None) and hasn't reached the target yet, the prediction is the remaining path.
	\item DONE (2) the agent is still in the grid (position is not None) but has already reached the target, the prediction is to stay at the target forever.
	\item DONE\_REMOVED (3) the agent has reached the target and it's removed from the grid.
\end{itemize}

Grid4TransitionsEnum extends Python IntEnum and assumes the following values:
\begin{itemize}
	\item NORTH (0)
	\item EAST (1)
	\item SOUTH (2)
	\item WEST (3)
\end{itemize}
Grid4TransitionsEnum is used to indicate absolute directions, related to the environment, like a compass.
Possible usages are storing where the agent is facing or computing legal actions, for example including as observation a one hot encoding of the directions where the agent can move.
\\
EnvAgent class models the agent and encapsulates in its internal state the following attributes:
\begin{itemize}
	\item initial\_position: Tuple[int, int], initial coordinate.
	\item initial\_direction: Grid4TransitionsEnum, the initial agent facing direction.
	\item direction: Grid4TransitionsEnum, the current facing direction.
	\item target: Tuple[int, int], the final coordinate.
	\item moving: bool, True if the agent is in a moving state.
	\item speed\_data: dictionary,
	\item malfunction\_data: dictionary,
	\item status: RailAgentStatus, the current agent status.
\end{itemize}

\subsection{The class RailEnv}

From the documentation
\begin{quotation}
	RailEnv is an environment inspired by a (simplified version of) a rail
    network, in which agents (trains) have to navigate to their target
    locations in the shortest time possible, while at the same time cooperating
    to avoid bottlenecks.
\end{quotation}

\subsubsection*{Environment Actions}
The avaiable actions are:
\begin{itemize}
	\item DO\_NOTHING (0) Default action if None has been provided or the value is not within this list. If agent.moving is True then the agent will MOVE\_FORWARD\@.
    \item MOVE\_LEFT (1) If agent.moving is False then becomes True. If it's possible turn the agent left, changing its direction, otherwise tries the action MOVE\_FORWARD\@.
    \item MOVE\_FORWARD (2) If agent.moving is False then becomes True. It updates the direction of the agent and if the new cell is a dead-end the new direction is the opposite of the current.
    \item MOVE\_RIGHT (3) If agent.moving is False then becomes True. If it's possible turn the agent right, changing its direction, otherwise tries the action MOVE\_FORWARD\@.
    \item STOP\_MOVING (4) If agent.moving is True then becomes False. A penalty will be added. Stop the agent in the current occupied cell.
\end{itemize}

\subsubsection*{Error Handling}

\subsubsection*{Malfunctions}
The strategy depends on the passed \textit{malfunction_generator_and_process_data}.

\subsubsection*{Rewards}

\newpage
%\bibliography{bibliography}
%\bibliographystyle{plain}

\end{document}